\documentclass[11pt,a4paper,twoside,openright]{report}

\usepackage[pdftex]{graphicx} % Biblioteca para uso de figuras
\usepackage{color}

\usepackage[brazil]{babel} % Biblioteca para uso da l�ngua portuguesa
\usepackage[T1]{fontenc} % Biblioteca para uso da acentua��o de entrada
\usepackage[latin1]{inputenc} % Biblioteca para uso da acentua��o de sa�da

\usepackage{amsthm,amsfonts,amsmath,amssymb}  % Biblioteca para uso de comandos matem�ticos
\usepackage{pslatex}
\usepackage{pstricks,pst-node,color,pst-gantt,pst-coil}
\usepackage{scalefnt}
\usepackage{float} %Permite colocar "\begin{figure}[H]" e colocar imagem exatamente onde desejar
\usepackage[hyphens]{url} % Para Aceitar URL nas referencias
\usepackage{pdfpages}

\usepackage{eurosym} %Pacote para possibilitar o uso do s�mbolo de euro "\euro"

\usepackage{listings} % para importa��o de c�digos fonte 

\usepackage[pdftex]{hyperref}
\hypersetup{bookmarks = true, % mostra a barra de bookmarks
	pdftitle = {O titulo do seu Trabalho}, % titulo
	pdfauthor = {Evandro L. L. Rodrigues}, % autor
	pdfsubject = {TCC}, % subject of the document
	pdfkeywords = {Sistemas Embarcados, ARM, Linux Embarcado}, % keywords
	colorlinks = true, 
	linkcolor = black,  
	citecolor = black,
	urlcolor = blue
}


\usepackage{comment}
\usepackage[margin=2.7cm]{geometry}
\renewcommand{\baselinestretch}{1.5}

\setlength{\parskip}{0em}

% Pacote para configurar cabe�alho e rodape
\usepackage{fancyhdr}
\pagestyle{empty}
\fancyhf{} % clear all header and footer fields

\fancypagestyle{plain}{\pagestyle{fancy}}
\renewcommand{\headrulewidth}{0pt}
\renewcommand{\footrulewidth}{0pt}


%Pacote para organizar apendices 
\usepackage[titletoc,toc]{appendix}
\renewcommand{\appendixtocname}{Ap�ndices}



\newenvironment{poliabstract}[1]
  {\renewcommand{\abstractname}{#1}\begin{abstract}}
  {\end{abstract}}

%%%%%%%%%%%%%%%%%%%%%%%%%%%%%%%%%%%%%% CONFIGURA��ES DE P�GINA %%%%%%%%%%%%%%%%%%%%%%%%%%%%%%%%%%%%%%
%\topmargin -2.1cm
%\oddsidemargin 0.5cm 
%\evensidemargin 0.5cm 
%\textwidth 15cm
%\textheight 25.1cm
	
%%%%%%%%%%%%%%%%%%%%%%%%%%%%%%%%%%%%%%%% IN�CIO DO DOCUMENTO %%%%%%%%%%%%%%%%%%%%%%%%%%%%%%%%%%%%%%%%
\begin{document}
	
%%%%%%%%%%%%%%%%%%%%%%%%%%%%%%%%%%%%%%  INCLUDES %%%%%%%%%%%%%%%%%%%%%%%%%%%%%%%%%%%%%%

%%%%%%%%%%%%%%%%%%%%%%%%%%%%%%%%%%%%%% ELEMENTOS PR�-TEXTUAIS %%%%%%%%%%%%%%%%%%%%%%%%%%%%%%%%%%%%%%
%%%%%%%%%%%%%%%%%%%%%%%%%%%%%%%%%%%%%%% CONFIGURA??ES DE CAPA %%%%%%%%%%%%%%%%%%%%%%%%%%%%%%%%%%%%%%%
\begin{titlepage}
	
	% CAPA PRINCIPAL
	\begin{center}
		\Huge{UNIVERSIDADE DE S�O PAULO}\\
		\vspace{0.02\textheight}
		\huge{ESCOLA DE ENGENHARIA DE S�O CARLOS}\\
		\vspace{0.01\textheight}
		\huge{DEPARTAMENTO DE ENGENHARIA EL�TRICA}\\
		\vspace{0.2\textheight}
		\huge{\textbf{T�tulo do Trabalho}}
		\vspace{0.2\textheight}
	\end{center}
		
		\large
		{
			\begin{flushleft}
			\Large{ \textbf{Autor}: \hspace{1cm} [Nome Aluno] }\\
			\Large{ \textbf{Orientador}: \hspace{0.3cm} Prof. Dr. Evandro Luis Linhari Rodrigues}\\
			\end{flushleft}
	
			\begin{center}
				\vspace{0.09\textheight}
				\Large{S�o Carlos}\\
				\Large{2017}
			\end{center}
		}
	
\end{titlepage}


%%%%%%%%%%%%%%%%%%%%%%%%%%%%%%%%%%%%%%%%%%%%%%% INSER??O P?GINA EM BRANCO %%%%%%%%%%%%%%%%%%%%%%%%%%%%%%%%%%%%%%%%%%%%%%
\cleardoublepage

%%%%%%%%%%%%%%%%%%%%%%%%%%%%%%%%%%%%%%% FOLHA DE ROSTO %%%%%%%%%%%%%%%%%%%%%%%%%%%%%%%%%%%%%%%
%\vspace{0.01\textheight} 
	\begin{center}
	\vspace{-0.06\textheight}
	%\thispagestyle{empty}
		\Large{\textbf{[Autor do Trabalho]}}\\
		\vspace{0.15\textheight}
		\Huge{\textbf{T�tulo do Trabalho}} 
		\vspace{0.08\textheight}
	\end{center}
		
		\large
		{
			\begin{flushright}
			\Large{Trabalho de Conclus�o de Curso apresentado} \hspace{1cm}\\
			\Large{� Escola de Engenharia de S�o Carlos, da}\\
			\Large{Universidade de S�o Paulo}\\
			\vspace{0.05\textheight}
			\Large{Curso de Engenharia El�trica}\\
			\vspace{0.05\textheight}
			\Large{ORIENTADOR: Prof. Dr. Evandro Luis Linhari Rodrigues}\\
			\end{flushright}
	
			\begin{center}
				\vspace{0.15\textheight}
				\Large{S�o Carlos}\\
				\Large{2017}
			\end{center}
		}



%%%%%%%%%%%%%%%%%%%%%%%%%%%%%%%%%%%%%%%%%%%%%%% FICHA CATALOGRAFICA % % % % % % % % % % % % % % % % % % %
\newpage

P�gina com a ficha catalogr�fica (em p�gina par).

%ficha catalografica no verso
%\includepdf{./Resources/ficha_catalografica.pdf}

% % % % % % % % % % % % % % % % % % % % % % % % Folha de aprova��o % % % % % % % % % % % % % % % % % % %
\newpage

p�gina com a folha de aprova��o (p�gina �mpar). \cleardoublepage

\begin{comment}
\begin{figure}[H]
	\centering
	\includegraphics[scale=0.3]{./Resources/aprovacao.jpg}
	\caption{Fluxo de comunica��o entre os principais componentes.}
	\label{Aprovacao}
\end{figure}
\cleardoublepage
\end{comment}

%%%%%%%%%%%%%%%%%%%%%%%%%%%%%%%%%%%%%%%%%%%%%%% DEDICAT?RIA %%%%%%%%%%%%%%%%%%%%%%%%%%%%%%%%%%%%%%%%%%%%%%
\
\vspace{0.11\textheight} 
\begin{center}
\textbf{\Huge{Dedicat�ria}}
\end{center}
\vspace{0.05\textheight}	
		
Texto dedicat�ria, texto dedicat�ria, texto dedicat�ria, texto dedicat�ria, texto dedicat�ria, Texto dedicat�ria, texto dedicat�ria, texto dedicat�ria, texto dedicat�ria, texto dedicat�ria, Texto dedicat�ria, texto dedicat�ria, texto dedicat�ria, texto dedicat�ria, texto dedicat�ria,Texto dedicat�ria, texto dedicat�ria, texto dedicat�ria, texto dedicat�ria, texto dedicat�ria.
		
\begin{flushright}
[Nome do Aluno].
\end{flushright}

%%%%%%%%%%%%%%%%%%%%%%%%%%%%%%%%%%%%%%%%%%%%%%% INSER��O P�GINA EM BRANCO %%%%%%%%%%%%%%%%%%%%%%%%%%%%%%%%%%%%%%%%%%%%%%
\cleardoublepage

%%%%%%%%%%%%%%%%%%%%%%%%%%%%%%%%%%%%%%%%%%%%%%% AGRADECIMENTOS %%%%%%%%%%%%%%%%%%%%%%%%%%%%%%%%%%%%%%%%%%%%%%
\
\vspace{0.11\textheight} 

\begin{center}
\textbf{\Huge{Agradecimentos}}
\end{center}

\vspace{0.05\textheight}
			
			
Texto agradecimentos, texto agradecimentos, texto agradecimentos, texto agradecimentos, texto agradecimentos, texto agradecimentos, texto agradecimentos, texto agradecimentos, texto agradecimentos, texto agradecimentos, texto agradecimentos, texto agradecimentos, texto agradecimentos, texto agradecimentos, texto agradecimentos.
Texto agradecimentos, texto agradecimentos, texto agradecimentos, texto agradecimentos, texto agradecimentos, texto agradecimentos, texto agradecimentos, texto agradecimentos, texto agradecimentos, texto agradecimentos, texto agradecimentos, texto agradecimentos.


\begin{flushright}
[Nome do Aluno].
\end{flushright}


%%%%%%%%%%%%%%%%%%%%%%%%%%%%%%%%%%%%%%%%%%%%%%% INSER��O P�GINA EM BRANCO %%%%%%%%%%%%%%%%%%%%%%%%%%%%%%%%%%%%%%%%%%%%%%
\cleardoublepage

%%%%%%%%%%%%%%%%%%%%%%%%%%%%%%%%%%%%%%%%%%%%%%% EP�GRAFE %%%%%%%%%%%%%%%%%%%%%%%%%%%%%%%%%%%%%%%%%%%%%%
\
\vspace{0.76\textheight} 

\begin{flushright}
\textit{"Ep�grafe ep�grafe ep�grafe ep�grafe}

\textit{Ep�grafe ep�grafe ep�grafe ep�grafe."}

[Autor Ep�grafe]
\end{flushright}


%%%%%%%%%%%%%%%%%%%%%%%%%%%%%%%%%%%%%%%%%%%%%%% INSER��O P�GINA EM BRANCO %%%%%%%%%%%%%%%%%%%%%%%%%%%%%%%%%%%%%%%%%%%%%%
\cleardoublepage

%%%%%%%%%%%%%%%%%%%%%%%%%%%%%%%%%%%%%%%%%%%%%%% RESUMO - PORTUGUES %%%%%%%%%%%%%%%%%%%%%%%%%%%%%%%%%%%%%%%%%%%%%%
\
\vspace{0.11\textheight} 

\begin{center}
\textbf{\Huge{Resumo}}
\end{center}

\vspace{0.05\textheight}
			
Resumo do trabalho, resumo do trabalho, resumo do trabalho, resumo do trabalho, resumo do trabalho, resumo do trabalho, resumo do trabalho, resumo do trabalho, resumo do trabalho, resumo do trabalho, resumo do trabalho, resumo do trabalho, resumo do trabalho, resumo do trabalho, resumo do trabalho, resumo do trabalho, resumo do trabalho, resumo do trabalho, resumo do trabalho.

\vspace{0.05\textheight}
	
Palavras-Chave: palavra1, palavra2, palavra3, palavra4, palavra5.

%%%%%%%%%%%%%%%%%%%%%%%%%%%%%%%%%%%%%%%%%%%%%%% INSER��O P�GINA EM BRANCO %%%%%%%%%%%%%%%%%%%%%%%%%%%%%%%%%%%%%%%%%%%%%%
\cleardoublepage

%%%%%%%%%%%%%%%%%%%%%%%%%%%%%%%%%%%%%%%%%%%%%%% RESUMO - INGL�S %%%%%%%%%%%%%%%%%%%%%%%%%%%%%%%%%%%%%%%%%%%%%%
\
\vspace{0.11\textheight} 

\begin{center}
\textbf{\Huge{Abstract}}
\end{center}

\vspace{0.05\textheight}	
		
Abstract, abstract, abstract, abstract, abstract, abstract, abstract, abstract, abstract, abstract, abstract, abstract, abstract, abstract, abstract, abstract, abstract, abstract, abstract, abstract, abstract, abstract, abstract, abstract, abstract, abstract, abstract, abstract, abstract, abstract, abstract, abstract, abstract, abstract, abstract, abstract, abstract, abstract, abstract, abstract, abstract, abstract, abstract.

\vspace{0.05\textheight}

Keywords: keyword1, keyword2, keyword3, keyword4, keyword5.

%%%%%%%%%%%%%%%%%%%%%%%%%%%%%%%%%%%%%%%%%%%%%%% INSER��O P�GINA EM BRANCO %%%%%%%%%%%%%%%%%%%%%%%%%%%%%%%%%%%%%%%%%%%%%%
\cleardoublepage
%\thispagestyle{empty}
%\newpage
%%%%%%%%%%%%%%%%%%%%%%%%%%%%%%%%%%%%%%%%%%%%%%% RESUMO %%%%%%%%%%%%%%%%%%%%%%%%%%%%%%%%%%%%%%%%%%%%%

%%%%%%%%%%%%%%%%%%%%%%%%%%%%%%%%%%%%% CONFIGURA��ES DOS �NDICES %%%%%%%%%%%%%%%%%%%%%%%%%%%%%%%%%%%%%
%\clearpage
%\thispagestyle{empty}
\listoffigures % ?ndice de Figuras
(Se houver...)
\listoftables % ?ndice de Tabelas
(Se houver...)
%%%%%%%%%%%%%%%%%%%%%%%%%%%%%%%%%%%%%%%%%%%%%%% INSER��O P�GINA EM BRANCO %%%%%%%%%%%%%%%%%%%%%%%%%%%%%%%%%%%%%%%%%%%%%%
\cleardoublepage

%%%%%%%%%%%%%%%%%%%%%%%%%%%%%%%%%%%%% LISTA DE ABREVIATURAS %%%%%%%%%%%%%%%%%%%%%%%%%%%%%%%%%%%%%
\
\vspace{0.11\textheight} 

\textbf{\Huge{Siglas}}\\
(Se houver...)
\vspace{0.05\textheight}
			
\begin{tabbing}
\hspace*{0.5cm}\=\hspace{2.5cm}\= \kill

% Exemplo de lista de lista de abreviaturas
\> MVC \> \textit{Model-View-Controller} - Modelo-Vis�o-Controlador \\
\> POO \> Programa��o Orientada a Objetos \\
\> UI \>  \textit{User Interface} - Interface do Usu�rio \\
\> UML \> \textit{Unified Modeling Language} - Linguagem de Modelagem Unificada \\


\end{tabbing}
\cleardoublepage
%%%%%%%%%%%%%%%%%%%%%%%%%%%%%%%%%%%%% CONFIGURA��ES DOS �NDICES %%%%%%%%%%%%%%%%%%%%%%%%%%%%%%%%%%%%%
%\usepackage{fancyhdr}
\pagestyle{fancy}
\fancyhf{} % clear all header and footer fields
\fancyhead[RO, LE] {\thepage}

\fancypagestyle{plain}{\pagestyle{fancy}}

\tableofcontents % �ndice Geral

%%%%%%%%%%%%%%%%%%%%%%%%%%%%%%%%%%%%%%%% ADI��O DOS CAP�TULOS %%%%%%%%%%%%%%%%%%%%%%%%%%%%%%%%%%%%%%%	
\chapter{Introdu��o}
\label{Introducao}

Introdu��o do trabalho.

Na reda��o da monografia, que � parte important�ssima do projeto, pois � aquela que ficar� p�blica, precisamos definir muito bem o t�tulo, construir um Resumo com todas as partes de um Resumo, esclarecer o(s) objetivo(s) e construir uma conclus�o completa, tudo isso quase que ao mesmo tempo, pois o trabalho j� terminou.
A Introdu��o deve conter um hist�rico (com muitas refer�ncias atuais-procure nas bases consagradas, como por exemplo IEEE, para apresentar n�o s� refer�ncias vol�teis-aquelas da Internet) do assunto apontando a origem e os avan�os que est�o publicados, ressaltando o "foco" de ataque do projeto.
Depois, compor um bom Embasamento Te�rico citando as refer�ncias de onde est� o assunto todo de cada t�pico (sem se confundir com a apresenta��o dos materiais usados no projeto), pois � o lugar onde estar�o presentes as partes da ci�ncia ou as t�cnicas de forma geral que foram utilizadas para construir a solu��o (exemplo: Sistemas Embarcados � um t�pico e n�o Raspberry Pi, Linux Embarcado � um t�pico e n�o a distribui��o escolhida, por�m Linux � um subt�pico de Sistemas Operacionais). Depois mostrar Material e M�todos, ou Desenvolvimento do Projeto, mostrando logo na entrada do cap�tulo uma figura ou diagrama que apresente de forma geral como as partes est�o relacionadas/conectadas. Mostrar os algoritmos da solu��o em forma de fluxogramas ou usando UML, de maneira clara e completa. Se for necess�rio apresentar trechos de c�digo para ressaltar solu��es ou apresentar abordagens, tamb�m que seja de forma direta e simplificada. C�digos completos dever�o estar nos Ap�ndices para a Monografia final, depois da defesa.  
Assim, consegue-se mostrar o quanto voc� evoluiu com os aprendizados do curso e com aqueles que voc� buscou a mais, e o quanto tem de solu��o legal e atual na sua proposta. 
Valorize as conquistas alcan�adas apresentando e discutindo os resultados com gr�ficos, tabelas, figuras, etc.
Nas Conclus�es, tamb�m n�o esque�a de apontar as defici�ncias ou limita��es que s�o inerentes ao trabalho e que no momento n�o est�o no(s) objetivo(s), e tamb�m as defici�ncias ou limita��es dos procedimentos que vc incorporou ao seu trabalho de outros autores.
Indique as dire��es para os trabalhos futuros, pois o autor conhece como ningu�m as oportunidades de continuidade e avan�os do trabalho.

Exemplo de cita��o de Refer�ncia \cite{referencia1}. Outra refer�ncia para a bibliografia \cite{referencia2}.

Refer�ncia para a figura \ref{logo}.

 \begin{figure}[H]
 	\centering
 	\includegraphics[scale=0.35]{./Resources/latex-logo.png}
 	\caption{Logo do LaTeX.}
 	\label{logo}
 \end{figure}



% % % % % % % % % % % % % % % % % % % % % % % % % % % % % % % % % % % % % % % % % % % % % % % % % % %
\section{Motiva��o}
Descrever a motiva��o do trabalho.


% % % % % % % % % % % % % % % % % % % % % % % % % % % % % % % % % % % % % % % % % % % % % % % % % % %
\section{Objetivo(s)}
Somente o(s) Objetivo(s) do trabalho.


% % % % % % % % % % % % % % % % % % % % % % % % % % % % % % % % % % % % % % % % % % % % % % % % % % %
\section {Justificativas/relev�ncia}
Justificativa do trabalho.


% % % % % % % % % % % % % % % % % % % % % % % % % % % % % % % % % % % % % % % % % % % % % % % % % % %
\section {Organiza��o do Trabalho}
Este trabalho est� distribu�do em XXX cap�tulos, incluindo esta introdu��o, dispostos conforme a descri��o que segue:

Cap�tulo 2: Descreve .....................................................................................

Cap�tulo 3: Discorre sobre .....................................................................................

Cap�tulo 4: Apresenta .....................................................................................

\chapter{Embasamento Te�rico ou Fundamenta��o Te�rica}
\label{EmbasamentoTeorico}

Embasamento te�rico para o desenvolvimento do trabalho.

Leia o texto que est� na Introdu��o e as dicas mais � frente...





\chapter{Material e M�todos}
\label{Materiais}


% % % % % % % % % % % % % % % % % % % % % % % % % % % % % % % % % % % % % % % % % % % % % % % % % % %
\section{Material}

Material utilizado no projeto.


% % % % % % % % % % % % % % % % % % % % % % % % % % % % % % % % % % % % % % % % % % % % % % % % % % %
\section{M�todos}

M�todos utilizados no projeto.




\chapter{Resultados e Discuss�es}
\label{Resultados}

Resultados e discuss�es sobre o trabalho.




\chapter{Conclus�o}
\label{Conclusao}

Conclus�es do trabalho de conclus�o de curso.

















%%%%%%%%%%%%%%%%%%%%%%%%%%%%%%%%%%%%%% ADI��O DAS REFER�NCIAS %%%%%%%%%%%%%%%%%%%%%%%%%%%%%%%%%%%%%	
\addcontentsline{toc}{chapter}{Refer�ncias}
\renewcommand{\bibname}{Refer�ncias}	
\bibliographystyle{unsrt} % Define o estilo da bibliografia
\bibliography{./Content/References} % Faz referencia ao arquivo ref.bib

%%%%%%%%%%%%%%%%%%%%%%%%%%%%%%%%%%%%%%%%%% ADI��O DOS AP�NDICES %%%%%%%%%%%%%%%%%%%%%%%%%%%%%%%%%%%%%%%%	
\begin{appendices}
%	\appendix
	\chapter{Cuidados e orienta��es para a elabora��o de texto}
\label{Ap�ndice Ap�ndice A}

\begin{center}
	\textbf {\color{blue}{Dicas para a reda��o de uma boa monografia de TCC}}
\end{center}

Observe as diretrizes no site do Depto.
\begin{center}
	\href{http://143.107.182.35/sel/files\_EE/tcc\_-\_diretrizes\_EESC\_v\_2010.pdf}{http://143.107.182.35/sel/files\_EE/tcc\_-\_diretrizes\_EESC\_v\_2010.pdf}
\end{center}

Veja no Portal de Livros Abertos da USP as mais novas vers�es das Diretrizes para apresenta��o de disserta��es e teses da USP.
\begin{itemize}
	\item Parte I (ABNT) - \href{http://dx.doi.org/10.11606/9788573140606}{http://dx.doi.org/10.11606/9788573140606}
	\item Parte II (APA) - \href{http://dx.doi.org/10.11606/9788573140576}{http://dx.doi.org/10.11606/9788573140576}
	\item Parte III (ISO) - \href{http://dx.doi.org/10.11606/9788573140590}{http://dx.doi.org/10.11606/9788573140590}
	\item Parte IV (Vancouver) - \href{http://dx.doi.org/10.11606/9788573140569}{http://dx.doi.org/10.11606/9788573140569}
\end{itemize}


Observe os elementos pr�-textuais neste documento....tem uma sequ�ncia a ser seguida (Capa, contracapa, Ficha catalogr�fica para a vers�o final, Listas de Figuras, Tabelas e S�mbolos/Abreviaturas).\\

\textbf{Resumo/abstract}\\
Texto em \underline{\textbf{um}} par�grafo apenas. Deve conter \underline{tudo} resumidamente (introdu��o, m�todo(s), resultados e conclus�es), de tal forma que seja poss�vel compreender a proposta e o que foi alcan�ado;\\
Palavras-chave: Logo abaixo do Resumo/Abstract.\\

\textbf{Cap�tulo 1} - Introdu��o\\
Realmente introduz o leitor indicando quais s�o as dire��es do trabalho ? Apresenta o tema e o objeto do trabalho e cont�m as Refer�ncias do Estado da arte (quem est� fazendo e em que n�vel os trabalhos da �rea est�o hoje)?\\
- Justificativa/relev�ncia do trabalho: explana��o sobre porque o trabalho se justifica e quais os pontos de relev�ncia do mesmo;\\
- Objetivo(s): \textbf{"\underline{somente}"} o(s) objetivo(s) em uma frase. Tamb�m podem ser descritos na forma de "gerais" \ e/ou "espec�ficos";\\
- Organiza��o do trabalho (o que tem em cada cap�tulo).\\

N�o h� necessidade de reproduzir (copiar) as obras que embasam o trabalho e sim colocar o suficiente para o entendimento do trabalho e citar as refer�ncias;

\textbf{Cap�tulo 2} - Embasamento Te�rico ou Fundamenta��o Te�rica\\
Revis�o da literatura dos t�picos  que sustentam a ci�ncia e o conhecimento, relativos ao(s) objetivo(s) e o(s) m�todo(s) escolhido(s) para o desenvolvimento do trabalho;

\textbf{Cap�tulo 3} - Material e M�todos ou Desenvolvimento do Projeto\\
Descri��o clara dos procedimentos e do material adotados para o desenvolvimento do trabalho (\underline{sem resultados}), incluindo sua adequa��o ao trabalho.\\ 
Tem que responder �s perguntas:\\
- Est� com tamanho adequado (proporcional) � monografia? \\
- H� informa��o suficiente e clara sobre os materiais e sobre os m�todos  adotados?\\
N�o h� necessidade de reproduzir (copiar) as obras que embasam o trabalho e sim colocar o suficiente para o entendimento do trabalho e citar as refer�ncias.

\textbf{Cap�tulo 4} - Resultados/Discuss�es\\
Aqui se mostra o que o trabalho permitiu produzir, e �s vezes o que pode ser comparado com outros trabalhos - aqui ficam claras se as propostas do trabalho s�o relevantes ou n�o, pois devem permitir a discuss�o do trabalho. 

Deve responder: Os resultados est�o claros em bom n�mero (nem muito nem pouco) que permitam avaliar realmente a proposta e o que foi produzido?

\textbf{Cap�tulo 5} - Conclus�es\\
"Fecham" com os objetivos? (respondem aos objetivos?) - aqui � que "se vende o peixe"  pois ir�o valorizar (ou n�o) o trabalho realizado. Normalmente � uma parte do trabalho "um pouco desprezada", pois o autor j� est� "cansado....". Mas aqui � um ponto importante de medida se o trabalho tem ou n�o valor.

\textbf{Refer�ncias}\\
Deve conter todas as refer�ncias {\color{red}{citadas no texto}}. Observar as Diretrizes, pois l� est�o os formatos corretos de cita��o.

\textbf{Ap�ndices}\\
Todo o material produzido pelo autor durante o trabalho, que o mesmo julga importante disponibilizar, mas que n�o deve estar no corpo do trabalho, pois atrapalharia a leitura do mesmo.

\textbf{Anexos}\\
Todo o material que n�o � de autoria pr�pria, mas que � importante para completar as informa��es do corpo do texto (ex. datasheet).\\



\begin{center}	
	\underline{Outras observa��es \textbf{IMPORTANTES} (\color{red}{leia isso com aten��o})}
\end{center}

NUNCA copie texto de outro autor sem a devida forma de cita��o (ver em diretrizes); a c�pia configura pl�gio! Com a Internet e/ou outras ferramentas dedicadas, � muito f�cil identificar se houve c�pia de texto.
Se voc� quiser verificar a porcentagem que seu texto apresenta de similaridade com outros na internet, baixe e rode o Copy Spider, por exemplo, ou consulte outros em  http://www.escritacientifica.sc.usp.br/anti-plagio/.
\begin{itemize}
	\item [$\Rightarrow$] O tempo verbal a ser usado no texto, de forma geral, � o "PASSADO", pois o trabalho j� aconteceu;
	\item [$\Rightarrow$] no texto, toda primeira vez que aparecer algum protocolo, procedimento, nome t�cnico, sigla, abreviatura, etc, al�m de explicar o que �, � necess�rio citar a refer�ncia. Exemplo: ...um girosc�pio (refer�ncia) � um tipo de sensor...
	\item [$\Rightarrow$] figura que n�o � de sua autoria deve conter a fonte;
	\item [$\Rightarrow$] capriche nas figuras (uma figura bem composta quase n�o precisa de texto para explic�-la);
	\item [$\Rightarrow$] todas as figuras e  tabelas devem ser referenciadas no texto;
	\item [$\Rightarrow$] procure manter a "\underline{Uniformidade de Nota��o}" para o texto todo, ou seja, se denominou ou se referiu a algo ou algu�m de uma certa forma, mantenha essa forma para se referir durante todo o texto;
	\item [$\Rightarrow$] n�o tenha medo de citar os trabalhos de outros autores (isso � imprescind�vel);
	\item [$\Rightarrow$] evite "muitas" refer�ncias de sites, pois s�o vol�teis - procure boas refer�ncias nas bases consagradas como a IEEE (http://ieeexplore.ieee.org/Xplore/home.jsp), pois possuem artigos de �timo n�vel;
	\item [$\Rightarrow$] {\color{red}{N�O USE O WIKIPEDIA COMO REFER�NCIA}};
	\item [$\Rightarrow$] todas as palavras escritas em ingl�s (ou em outras l�nguas) devem estar em it�lico;
	\item [$\Rightarrow$] cuidado com o uso de "atrav�s", que significa "atravessar" algo e n�o por meio de ;
	\item [$\Rightarrow$] todas as obras citadas nas refer�ncias devem estar citadas no texto;
	\item [$\Rightarrow$] evite o uso de "satisfat�rio", "razo�vel" ou outra palavra que n�o seja precisa ou que n�o tenha sido definida a ordem de grandeza no texto;
	\item [$\Rightarrow$] c�digos de programas devem estar em Ap�ndices, pois servem para comprovar o desenvolvimento e facilitar a reprodu��o do trabalho;
	\item [$\Rightarrow$] Anexos s�o informa��es que n�o s�o de sua autoria, mas que s�o importantes e que devem fazer parte da monografia para auxiliar e esclarecer o leitor.
\end{itemize} 



% % % % % % % % % % % % % % % % % % % % % % % % % % % % % % % % % % % % % % % % % % % % % % % % % % %
\chapter{Apresenta��o do TCC}
\label{Ap�ndice  Ap�ndice B}

\begin{center}
	\textbf {\color{blue}{Cuidados e orienta��es para a elabora��o da Apresenta��o do TCC}}
\end{center}


{\color{red}{Todos os meus alunos me enviam a apresenta��o previamente, pois faz parte do procedimento que adoto para os TCCs.}}\\
Como tem-se at� 30 minutos para fazer a apresenta��o deve-se dimensionar a quantidade de slides para isso. Cada um tem seu "timming" com rela��o � quantidade de informa��o versus tempo dispon�vel para apresenta��o.
Os slides devem ser sempre muito mais visuais que textuais, ou seja, n�o se deve colocar frases e "ficar lendo" as mesmas. Os slides devem apresentar uma forma "clean" para que sirvam apenas de guia para a apresenta��o do trabalho. 

Leia no site da El�trica (/Gradua��o/Trabalhos de Conclus�o de Curso - TCC) as DIRETRIZES GERAIS PARA ELABORA��O DO TRABALHO DE FORMATURA - TCC,
onde pode-se encontrar as Fichas de Avalia��o que s�o sugeridas pelo Depto, que n�o s�o necessariamente seguidas � risca pelos avaliadores, mas que servem de bom guia para os alunos entenderem como s�o feitas as avalia��es.
Tente n�o utilizar fundo escuro, pois escurece o ambiente e �s vezes n�o se consegue o visual esperado. Sempre que poss�vel teste antes no local da apresenta��o. \\
Resumindo:
\begin{itemize}
	\item [$\Rightarrow$]Prepare o seu ambiente de apresenta��o - mesa, cadeiras, etc., colocadas de maneira a n�o te atrapalhar;
	\item [$\Rightarrow$]Teste as cores que o projetor realmente projeta para que a visualiza��o seja pr�xima daquela constru�da nos slides;
	\item [$\Rightarrow$]Evite usar fundo escuro;
	\item [$\Rightarrow$]A apresenta��o deve dedicar o maior tempo para o trabalho em si, suas propostas, seus resultados/discuss�es e conclus�es;
	\item [$\Rightarrow$]Coloque um Sum�rio resumido da apresenta��o e n�o do trabalho todo;
	\item [$\Rightarrow$]Descarregue os slides de textos excessivos - os slides devem servir de guia para a apresenta��o e suporte visual para o p�blico;
	\item [$\Rightarrow$]Slides com numera��o - facilita o controle e a identifica��o do conte�do;
	\item [$\Rightarrow$]Frases longas dificultam a apresenta��o pois induz o p�blico � leitura e n�o � apresenta��o do palestrante; 
	\item [$\Rightarrow$]Sua apresenta��o deve "caber" dentro de 15 a 30 minutos;
	\item [$\Rightarrow$]N�o colocamos slides sobre Refer�ncias na apresenta��o, a menos que alguma(s) publica��o(�es) seja(m) muito importante(s) a ponto de merecer destaque na apresenta��o;
	\item [$\Rightarrow$]O �ltimo slide deve conter \textbf{"\underline{OBRIGADO}"} e n�o "Perguntas". 
	\item [$\Rightarrow$]Como sou o orientador, eu serei o condutor de todo o ritual da defesa.
	
\end{itemize}


% % % % % % % % % % % % % % % % % % % % % % % % % % % % % % % % % % % % % % % % % % % % % % % % % % %
\chapter{Monografia Parcial do TCC}
\label{Ap�ndice  Ap�ndice C}


\begin{center}
	\textbf {\color{blue}{Cuidados e orienta��es para a composi��o da Monografia Parcial do TCC}}
\end{center}


Trata-se de uma Monografia completa, com todas as partes de uma Monografia final. 

Atente-se para as partes em {\color{red}{vermelho}}.

\begin{itemize}
	\item Resumo
	\item Introdu��o
	\item Objetivos
	\item Justificativas/Relev�ncia
	\item Embasamento Te�rico (Fundamenta��o Te�rica-Revis�o Bibliogr�fica)
	\item Material e m�todos ou Desenvolvimento do Projeto
	\item {\color{red}{Resultados Preliminares}}
	\item {\color{red}{Conclus�es Preliminares}}
	\item {\color{red}{Sequ�ncia do trabalho (indicando poss�veis corre��es de rota do projeto)}}
	\item {\color{red}{Cronograma Final (com corre��es se necess�rio)}}
	\item Refer�ncias 
	\item Ap�ndices
	\item Anexos
\end{itemize} 

Sendo bem feito, ir� poupar esfor�o para a reda��o da monografia.


\end{appendices}

%%%%%%%%%%%%%%%%%%%%%%%%%%%%%%%%%%%%%% CONFIGURA��ES DE ANEXOS %%%%%%%%%%%%%%%%%%%%%%%%%%%%%%%%%%%%%%
\newcommand{\annexname}{Anexo}
\makeatletter
\newcommand\annex{\par
	\setcounter{chapter}{0}%
	\setcounter{section}{0}%
	\gdef\@chapapp{\annexname}%
	\gdef\thechapter{\@Roman\c@chapter}}
\makeatother



\annex
\addcontentsline{toc}{chapter}{Anexos}
\input{./Content/8_Anexos.tex}

%%%%%%%%%%%%%%%%%%%%%%%%%%%%%%%%%%%%%%% T�RMINO DO DOCUMENTO %%%%%%%%%%%%%%%%%%%%%%%%%%%%%%%%%%%%%%%%
\end{document}    
	